\chapter{Conclusion}
\label{capitulo6}


In the next years, autonomous vehicles will be a new reality, and this moment the topics regarding object detection and machine learning are being the hot trends in the computer science domain. And with this, it is necessary to improve the methods of computer vision to improve this related technology. It is possible to apply this algorithm to detect other kinds of objects and perform distance measurement as well. 

A Real-time distance measurement method with multi-cameras for object detection on the roads is introduced in this work. The utilized method is based on using multi-cameras, which are two cameras mounted in the same horizontal position and displaced vertically by a predefined distance (the base). To measure the distance to objects, a vehicle detection method is performed first following two steps: hypothesis generation and hypothesis verification. 

% Several methods apply the detection task on both images, which is time-consuming. However, in this paper the vehicles are detected first in only one camera then similar vehicles are detected on the other camera using a stereo matching method. After detecting and matching the same vehicles in both cameras, the distance measurement method based on the distance between the two cameras, the position of vehicles in both cameras and certain geometric angles, is performed. Although the method is based on a relatively simple algorithm, the distance is measured accurately. Furthermore, a comparison between the proposed method and some other methods from literature was performed to evaluate the proposed method, where it showed that despite the simplicity of the proposed method, it measures the distance with high accuracy. The proposed method may be used to perform several tasks in several systems such as computing safety distance between vehicles and vehicle speed and it may also be used to measure objects distances regardless of their types by simply changing the detection algorithm.

This work also makes a comparative approach between the other several state-of-the-art techniques algorithms to perform distance measurement to choose the best and faster technique. 

One of the big problems on camera arrays is the moment to put in the correct angle and high level, due to this a small filter to clear this threshold is necessary to be implemented and remove these noises. The part of calibration in the proposed technique 1 is a little bit difficult, because it is necessary to remember the tool error and the measurement error, and with this is possible to calibrate in the wrong way. 

It was also proved that the multi-cameras perspective combined with a high order mathematical technique is better for this work, where it is possible to divide the work of the detection into two different or more cameras. However for our motivation and based on the current literature the angle between the camera is +30 degrees, with this will allow us to cover a big part of the scenario.   

\section{Future works}

According to all results collected in this work, it is possible to suggest the following approaches:

\begin{itemize}
    \item Collect data and label data to perform predictions for a specific scenario.
    \item Combine data from multiple sensors, such as LIDAR or RADAR using data fusion approaches to increase the accuracy of the measurement. 
    \item Apply other kinds of algorithms, like EfficienceDet or other novel algorithms for object detection and distance measurement as well.
    \item Use the proposed test in the real scenario because the algorithm was just performed in controlled sites. 
    \item Perform other mathematical approaches to reduce the dimensionality of the data.
\end{itemize}