%TCIDATA{LaTeXparent=0,0,relatorio.tex}

\resumo{Abstract}{Autonomous Vehicles can reduce the number of car crashes and the number of fatal victims. Following the German Statistical Department, just in 2019, there were over 2 million car accidents, and more than 90 percent of crashes are caused by human errors (National Highway Traffic Safety Administration, 2015). This work proposes a multi-camera system for object detection and distance measurement using computer vision that it will support some autonomous vehicle tests and improve safety during the tests. Three approaches are performed and compared with the real distance, and just the best technique was included in the proposed framework. In most cases, this error is directly related to meteorological factors and weak communication signals between cameras and the control hardware. The results obtained show that object detection methods guarantee precision with an accuracy above 93~\% in ideal conditions and controlled environments. However, accuracy is reduced when obstacles are present in front of the detected object. Additional techniques are also proposed to optimize the positioning of the cameras and the angle of inclination.}

\vspace*{2cm}

\resumo{Resumo}{Os veículos autônomos podem reduzir o número de acidentes automobilísticos e o número de vítimas fatais. Segundo o Departamento de Estatística da Alemanha, apenas em 2019, ocorreram mais de 2 milhões de acidentes de carro. Este trabalho propõe um sistema multicâmera para detecção de objetos e medição de distâncias por visão computacional que irá apoiar alguns testes de veículos autônomos e melhorar a segurança durante os testes. Três abordagens são realizadas e comparadas com a distância real, e apenas a melhor técnica foi incluída no framework proposto. Na maioria dos casos, esse erro está diretamente relacionado a fatores meteorológicos e sinais de comunicação fracos entre as câmeras e o hardware de controle. Os resultados obtidos mostram que os métodos de detecção de objetos garantem precisão com exatidão acima de 93 ~ \% em condições ideais e ambientes controlados. No entanto, a precisão é reduzida quando os obstáculos estão presentes na frente do objeto detectado. Técnicas adicionais também são propostas para otimizar o posicionamento das câmeras e o ângulo de inclinação.}

