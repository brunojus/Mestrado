%TCIDATA{LaTeXparent=0,0,relatorio.tex}

\resumo{Abstract}{This work proposes a multi-camera system for object detection and distance measurement using computer vision. Three approaches are performed and compared with the real distance, and just the best technique was included in the proposed framework. For the algorithms implemented, several different scenarios and characteristics are taken into account for estimating distance measurement error. In most cases, this error is directly related to meteorological factors and weak communication signals between cameras and the control hardware. The results obtained show that, in ideal conditions and controlled environments, object detection methods guarantee precision with an accuracy above 93~\%. However, accuracy is drastically reduced when obstacles are present in front of the detected object. Additional techniques are also proposed to optimize the positioning of the cameras and the angle of inclination.}

\vspace*{2cm}

\resumo{Resumo}{Este trabalho propõe um sistema multicâmera para detecção de objetos e medição de distâncias usando visão computacional. Três abordagens são implementadas e comparadas com a distância real, e apenas a melhor técnica foi incluída no framework proposto. Para os algoritmos implementados, vários cenários e características diferentes são levados em consideração para estimar o erro de medição de distância. Na maioria dos casos, esse erro está diretamente relacionado a fatores meteorológicos e sinais de comunicação fracos entre as câmeras e o hardware de controle. Os resultados obtidos mostram que, em condições ideais e ambientes controlados, os métodos de detecção de objetos garantem precisão com exatidão acima de 93 ~ \%. No entanto, a precisão é drasticamente reduzida quando os obstáculos estão presentes na frente do objeto detectado. Técnicas adicionais também são propostas para otimizar o posicionamento das câmeras e o ângulo de inclinação.}

