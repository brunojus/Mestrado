\begin{flushright}
\chapter*{Resumo}
\end{flushright}

\noindent
Neste trabalho é realizada a proposta de um sistema multicameras para detecção de objetos, e também a medição da distância através de visão computacional. Avalia-se o desempenho de outras técnicas para levar em consideração o melhor resultado do framework proposto. Uma vez implementados os algoritmos, existem diversos outros cenários e caractéristicas que podem ser levadas em consideração para um possível erro de medição da distância, na maioria das vezes esse erro está diretamente relacionado a fatores meteriológicos e também a sinal fraco de comunicação entre as câmeras e o hardware de controle. Através dos resultados obtidos, percebe-se que em condições ideias e em ambientes controlados, os métodos de detecção de objetos garatem precisão satisfatória, com a acurácia acima de 93\%, mas quando existe outro objeto em frente a ele, a acurácia reduz drásticamente. Para reduzir esses problemas, foi proposto a otimização do posicionamento das câmeras, assim como o ângulo de inclinação. 


\noindent
Palavras-chave: \textbf{Veículos autônomos; Visão computacional; Detecção de objetos; Estimação de distância}.



