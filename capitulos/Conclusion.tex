\chapter{Conclusion}
\label{capitulo6}


In the next years, autonomous vehicles will be a new reality, and this moment the topics regarding object detection and machine learning are the hot trends in the computer science domain. Furthermore, with this, it is necessary to improve computer vision methods to strengthen this related technology. Applying this algorithm to detect other classes of objects and perform distance measurement is possible. 

A Real-time distance measurement method with multi-cameras for object detection on the roads is introduced in this work. The utilized method is based on using multi-cameras, which are two cameras mounted in the same horizontal position and displaced vertically by a predefined distance (the base). A vehicle detection method is performed first following two steps: hypothesis generation and hypothesis verification. 


This work also compares several state-of-the-art techniques algorithms to perform distance measurement to choose the best and faster technique. 

One of the big problems on camera arrays is the moment to put in the correct angle and high level, due to this a small filter to clear this threshold is necessary to be implemented and remove these noises. The part of calibration in the proposed technique 1 is a little bit difficult because it is needed to remember the tool error and the measurement error, and with this is possible to calibrate in the wrong way. 

It was also proved that the multi-cameras perspective combined with a high order mathematical technique is better for this work, where it is possible to divide the work of the detection into two different or more cameras. However, for our motivation and based on the current literature, the camera angle is +30 degrees, which will allow us to cover a big part of the scenario.   

\section{Future works}

According to all results collected in this work, it is possible to suggest the following approaches:

\begin{itemize}
    \item Collect data and label data to perform predictions for a specific scenario.
    \item Combine data from multiple sensors, such as LIDAR or RADAR, using data fusion approaches to increase the measurement's accuracy. 
    \item Apply other algorithms, like EfficienceDet or other novel algorithms for object detection and distance measurement.
    \item Use the proposed test in the real scenario because the algorithm was just performed in controlled sites. 
    \item Perform other mathematical approaches to reduce the dimensionality of the data.
\end{itemize}