\documentclass[a4paper,12pt,openright,titlepage,oneside]{book}

%\usepackage[english,brazil]{babel} 
%Define regra de gramática para separar síbalas {babel}
%e altera os títulos (como chapters, sections, references) para português
\usepackage[english]{babel}
\usepackage[hidelinks]{hyperref}
\usepackage{listings}
\usepackage{titletoc}
\usepackage{notoccite}
\usepackage{float}
\usepackage{algorithm}
%\usepackage[algo2e]{algorithm2e} 
\usepackage{arevmath}     % For math symbols
\usepackage[noend]{algpseudocode}
% Define uso de caracteres acentuados no PDF gerado
% e permite copiar corretamente o texto do PDF.
\usepackage[T1]{fontenc}
\usepackage[utf8]{inputenc}
\usepackage{booktabs}
\usepackage{mathtools}
\usepackage{amsmath}
\usepackage{csvsimple}
\usepackage{appendix}

\makeatletter
\AtBeginDocument{\let\book@l@chapter\l@chapter}
\newcommand{\demotechaptersintoc}{%
  \addtocontents{toc}{\let\protect\l@chapter\protect\l@section}%
}
\newcommand{\promotechaptersintoc}{%
  \addtocontents{toc}{\let\protect\l@chapter\protect\book@l@chapter}%
}


\usepackage{doi}
% \usepackage[natbibapa]{apacite}
% Adição do pacote do template da UnB
\usepackage{template-FT-UnB/ft2unb}

\DeclareGraphicsExtensions{.jpg,.pdf,.mps,.png,.gif, .eps}
\graphicspath{imagens} %define diretório de imagens

%Arquivo com lista com hifenização correta de algumas palavras.
%Defina novas palavras no arquivo a medida que verificar que a hifenização automática
%etá errada para tais palavras.
\input{hifenizacao} 
\usepackage{xcolor}

% \newcommand*{\doi}[1]{\href{http://dx.doi.org/#1}{doi: #1}}
% \newcommand{\todo}[1]{\textcolor{red}{TODO: #1}\PackageWarning{TODO:}{#1!}}

% \usepackage[num,abnt-emphasize=bf,abnt-etal-text=it,abnt-doi=doi,bibjustif]%
% % ALTERE OS VALORES DENTRO DAS CHAVES DOS COMANDOS NESTA SEÇÃO PARA INCLUIR OS SEUS DADOS E DADOS DA SUA
% % DISSERTAÇÃO DE MESTRADO OU TESE DE DOUTORADO
% % -----------------------------------------------------------------------------------------------------
% }
%\onehalfspacing
\title{Framework for Object Detection and Distance Measurement using Multi-Cameras Perspective}

\author{BRUNO JUSTINO GARCIA PRACIANO}
\date{2020-07-16} %data da defesa

\grau{Mestre} %Mestre ou Doutor
\area{Sistemas Mecatrônicos} %Nome do curso
\siglaarea{ENM} %Sigla do departamento
\tipodemonografia{Dissertação} %Dissertação ou Tese
\programa{Mestrado} %Mestrado ou Doutorado
\autorendereco{} %Endereço do autor da dissertação/tese
\totalpgs{140} %total de páginas atualmente na sua dissertação
\dia{16} %dia da defesa
\mes{Julho} %mês da defesa
\ano{2020} %ano da defesa
\numpublicacao{150/2019} %número da publicação (após a defesa, tal número deve ser obtido na secretaria)

%PPGEE.DM  = Programa de Pós Graduação em ENgenharia Elétrica.Dissertação de Mestrado
%PPGENE.TD  = Programa de Pós Graduação em ENgenharia Elétrica.Tese de Doutorado
\siglapublicacao{PPMEC.TD}

\titulolinhai{Framework for Object Detection and Distance }
\titulolinhaii{Measurement using Multi-Cameras Perspective}


\autori{BRUNO JUSTINO GARCIA PRACIANO}
%Caso seu nome não caiba em uma única linha, divida ele nos comandos abaixo
%\autorii{} 
%\autoriii{}

\membrodabancai{Prof. Dr.-Ing João Paulo Carvalho Lustosa da Costa, ENE/UnB}
\membrodabancaifuncao{Membro Interno - Presidente}
\membrodabancaii{Prof. Dr. Rafael T. de Sousa Jr., ENE/UnB}
\membrodabancaiifuncao{Examinador Interno}
\membrodabancaiii{Prof. Dr. Robson de Oliveira Albuquerque, CEPESC}
\membrodabancaiiifuncao{Examinador Externo}




% -----------------------------------------------------------------------------------------------------

%line-numbers, inputencoding=utf8/latin1
%Define o estilo para listagens de código fonte
\lstset{
  numbers=left, %numeração de linhas à esquerda
  stepnumber=1,
  firstnumber=1,
  numberstyle=\tiny,
  extendedchars=true,
  frame=none,
  basicstyle=\footnotesize,
  stringstyle=\ttfamily,
  showstringspaces=false,
  %language=Java, %deve ser definida na inclusão de cada trecho de código, pois podem existir linguagens diferentes em exemplos diferentes
  breaklines=true,
  breakautoindent=true,
  %estilos de comentário de uma e várias linhas
  morecomment=[l]{--}, morecomment=[s]{/*}{*/}, morecomment=[s]{<!--}{-->}, morecomment=[s]{--[[}{--]]}
}

% Adição de metadados no PDF (propriedades do documento PDF)
\makeatletter
	 \hypersetup{
		 pdftoolbar=true,        % show Acrobat’s toolbar?
		 pdfmenubar=true,        % show Acrobat’s menu?
		 pdffitwindow=false,     % window fit to page when opened
		 pdfstartview={FitH},    % fits the width of the page to the window	 
		 pdftitle={\@title},
		 pdfauthor={\@author},
		 pdfsubject={\tipodemonografianome \ de\ \programastr \ em\ \areastr},   % subject of the document
		 pdfcreationdate={\pdfdate}
	 }
\makeatother


\makeindex
\makenomenclature %Necessário para gerar lista de siglas


\begin{document}

	\pdfbookmark[0]{Agradecimentos}{agradecimentos}
	%* indica para nao adicionar numeracao ao titulo
\chapter*{Acknowledgments}

First of all, I would like to thank God for giving me health and concentration during the COVID19 crises in Europe and for not allowing anything bad to happen with me.

Also, I leave my immense gratitude for the University of Brasilia for the acquired knowledge along these years.


I am eternally thankful to my family Alessandra Justino, Flávio Praciano, Lenita Justino, and Victor Hugo, whom I can count and trust unconditionally.

An invaluable character is my master tutor Prof. Dr.-Ing. Joao Paulo C. L da Costa, who directed me to the new field of autonomous vehicles, and for always motivate me to do my best.

I would like to demonstrate my gratitude for my advisor Lothar Weichenberger for trust in my work and always support me and a special thanks for the company Elektronische Fahrwerksystem GmbH for financial and structural support. Another very essential characters in my thesis are Lukac Branimir, Tobias Behn, and Andreas Schustek without their valuable support it would not possible to do much of my work.

Along this road, I had some problems with german bureaucracy, my sincere thanks for Lukas Bös for his unbelievable capacity to solve problems and his communication skills, also I need to thank Sara Martin and Robin Käsmayr for their support and also students who attended our Stammtisch. And Vanessa Voll for all the help provided.

I would like to thanks the lunch clubs of Latino brothers composed by Javier Rivas, Arnaldo Arancíbia, Gabriel Pinheiro, João Paulo, and me for the good conversations and for supporting me to improve my Spanish while I was living in Germany.

My time in Germany was excellent due to the great hospitality of my host family, my thankful to Johanna Hirschmann and Herbert Hirschmann for made it easy everything. 

Additionally, his unconditional helpfulness and psychological support were simply
crucial, my friends were very important, in special my flatmate Gabriel Pinheiro, Lucas Maciel, Yan Trindade, Fábio Mendonça, Daniel Alves, Francisco Lopes, Robson de Albuquerque, and Rafael Timoteo. 

I also express my sincere gratitude for the professors from the University of Brasilia in special for Prof.Dr. Rafael Timoteo de Sousa Junior for his invaluable support and his guidance along with my career. And for prof. Dr. Edna Canedo, prof. Dr. Georges Amvame for partnership in some papers publications.

My sincere gratitude for the Institutional Security Office of the Presidency of the Republic of Brazil (Grant 002/2017) and to FAPDF (Projetos UIoT 0193.001366/2016) whose I have a scholarship and supported me to pay the conference fees. 
	\begin{flushright}
\chapter*{Resumo}
\end{flushright}

\noindent
Neste trabalho é realizada a proposta de um sistema multicameras para detecção de objetos, e também a medição da distância através de visão computacional. Avalia-se o desempenho de outras técnicas para levar em consideração o melhor resultado do framework proposto. Uma vez implementados os algoritmos, existem diversos outros cenários e caractéristicas que podem ser levadas em consideração para um possível erro de medição da distância, na maioria das vezes esse erro está diretamente relacionado a fatores meteriológicos e também a sinal fraco de comunicação entre as câmeras e o hardware de controle. Através dos resultados obtidos, percebe-se que em condições ideias e em ambientes controlados, os métodos de detecção de objetos garatem precisão satisfatória, com a acurácia acima de 93\%, mas quando existe outro objeto em frente a ele, a acurácia reduz drásticamente. Para reduzir esses problemas, foi proposto a otimização do posicionamento das câmeras, assim como o ângulo de inclinação. 


\noindent
Palavras-chave: \textbf{Veículos autônomos; Visão computacional; Detecção de objetos; Estimação de distância}.




	

\noindent
\lipsum[5]
\noindent
Palavras-chave: \textbf{Internet das coisas (IoT); \textit{middleware} IoT; Federação IoT; Serviços \textit{peer-to-peer} (P2P)}.




	\pdfbookmark[0]{Sumário}{sumario}
	\sumario
	
	\pdfbookmark[0]{List of Figures}{listafiguras}
	\listadefiguras
	
	\pdfbookmark[0]{List of Tables}{listatabelas}
	\listadetabelas
	
% 	\pdfbookmark[0]{Lista de Códigos Fonte}{listacodigosfonte}
% 	\listadecodigosfonte
	
	\renewcommand{\nomname}{ACRONYMS} %Define um caption à lista de siglas
	%Inclui a lista de siglas 
	\pdfbookmark[0]{ACRONYMS}{nomenclatura}
	
	\nomenclature{AV}{Autonomous Vehicles}
	\nomenclature{ML}{Machine Learning}
	\nomenclature{ANN}{Artificial Neural Network}
	\nomenclature{CNN}{Convolutional Neural Network}
	\nomenclature{WIFI}{Wireless Network Protocol}
	\nomenclature{IDE}{Integrated Development Environment}
	\nomenclature{IP}{Internet Protocol Version 4}
	\nomenclature{JSON}{JavaScript Object Notation}
	\nomenclature{REST}{Representational State Transfer}
	\nomenclature{SQL}{Structured Query Language}
	\nomenclature{TCP}{Transmission Control Protocol}
	\nomenclature{API}{Application Programming Interface}
	\nomenclature{WSN}{Wireless Sensor Networks}
	\nomenclature{GPS}{Global Positioning System}
    \nomenclature{LIDAR}{Light Detection and Ranging}
    \nomenclature{RADAR}{RAdio Detection And Ranging}
	\nomenclature{VM}{Virtual Machine}
	\nomenclature{IEEE}{Institute of Electrical and Electronics Engineers}
	

	
	\printnomenclature[2.5cm] 
	\mainmatter %Inicia a numeracao normal cardinal
	\setcounter{page}{1} \pagenumbering{arabic} \pagestyle{plain}
	
	\chapter{Introduction} \label{introducao}
In this chapter the problem statement will be defined. 
\section{Motivation}

The autonomous vehicles are the new reality for the next years with this premise, the new solutions need to grow, and one of these important necessity is the vehicle tracking. Given that, there are many levels of self-drive car automation as shown in Figure \ref{fig:automation}. This work is focused on test scenario of level 5 of driving automation. Due to drones requires many battery during the test and limits the scenario to 30 to 45 minutes, the proposal is to use the infrastructere along the road as camera on the pole and send the data to a command central.

\begin{figure}[H]
\centering
\includegraphics[scale=0.8]{imagens/automation.jpg}
\caption{Refazer figura}
\label{fig:automation}
\end{figure}

\section{Problem Description}

The problematic of this theme is to create an architecture to detect objects along the roads combined with the distance of these along the road. The first approach could be using a drone, but due to the battery conditions for a long tests it is not a good choice. Figure \ref{fig:tests} shown this problem. The goal is to track the position of the cars even vehicle under the test (VUT) and traffic simulation data (TSV) along the road and send this information to the command central (CC). 

\begin{figure}[H]
\centering
\includegraphics[scale=0.5]{imagens/proposal.png}
\caption{Modelling of the project using drones, where CC means command central, VUT is vehicle under the tests and TSV is traffic simulation vehicle.}
\label{fig:tests}
\end{figure}

\section{Objectives}

Create a framework able to detect objects as cars, trucks, motorcycles, pedestrians and other strange obstacles along the road, and with these capacities estimate the distance of this object along the road, this framework is necessary to work with a camera array.

\section{Published Works}

Along the development of this work, the author worked in several areas of computer sciences, in special data science domain and always trying to keep the 
multidisciplinarity. These are my published works along my master degree:

\begin{enumerate}

\item \textbf{PRACIANO, BRUNO J. G.}; DE CALDAS FILHO, FRANCISCO L. ; E MARTINS, LUCAS M. C. ; DA CUNHA, DAYANNE F. ; DA SILVA, DANIEL ALVES ; DE SOUSA JÚNIOR, RAFAEL TIMÓTEO. SEGURANÇA DO AMBIENTE USANDO DISPOSITIVO IOT COM PROCESSAMENTO DISTRIBUÍDO. In: Atas da conferência IberoAmericana WWW/Internet 2019, 2019. Atas da conferência Ibero-Americana WWW/Internet 2019, 2019. p. 163

\item MARQUES, Angelica Alves da Cunha; \textbf{PRACIANO, Bruno Justino Garcia}. Researchers of the Brazilian archivistics scientific community in French international areas of interlocution Encontros Bibli: revista eletrônica de biblioteconomia e ciência da informação, Florianópolis, v. 25, p. 01-14, mar. 2020. ISSN 1518-2924. doi:https://doi.org/10.5007/1518-2924.2020.e65864.

\item CASTELINO, R. M. ; MOREIRA, G. P. ; \textbf{PRACIANO, BRUNO JUSTINO GARCIA} ; SANTOS, GIOVANNI A. ; WEICHENBERGER, L. ; DE SOUSA, JR, RAFAEL T. . Improving the accuracy of pedestrian detection in partially occluded or obstructed scenarios. In: 2020 10th International Conference on Advanced Computer Information Technologies, 2020, Deggendorf. 2020 10th International Conference on Advanced Computer Information Technologies, 2020. (to be published)

\item CANEDO, EDNA ; PINHEIRO, GABRIEL ; SOUSA JR., RAFAEL ; RIBEIRO, RENATO ; \textbf{PRACIANO, BRUNO} ; LOPES DE MENDONÇA, FÁBIO . Front End Application Security: Proposal for a New Approach. In: 22nd International Conference on Enterprise Information Systems, 2020, Prague. Proceedings of the 22nd International Conference on Enterprise Information Systems, 2020. p. 233.

\item SOUSA JR., RAFAEL ; LOPES DE MENDONÇA, FÁBIO ; NZE, GEORGES ; PINHEIRO, GABRIEL ; \textbf{PRACIANO, BRUNO }; CANEDO, EDNA . Performance Evaluation of Software Defined Network Controllers. In: 10th International Conference on Cloud Computing and Services Science, 2020, Prague. Proceedings of the 10th International Conference on Cloud Computing and Services Science, 2020. p. 363.


\item SILVA, DANIEL ALVES DA ; TORRES, JOSÉ ALBERTO SOUSA ; PINHEIRO, ALEXANDRE ; DE CALDAS FILHO, FRANCISCO L. ; MENDONÇA, FABIO L. L. ; \textbf{PRACIANO, BRUNO J. G} ; KFOURI, GUILHERME OLIVEIRA ; DE SOUSA, JR, RAFAEL T. . Inference of driver behavior using correlated IoT data from the vehicle telemetry and the driver mobile phone. In: 2019 Federated Conference on Computer Science and Information Systems, 2019. org.crossref.xschema.\_1.Title@7d70270b, 2019. p. 487.

\item KFOURI, GUILHERME DE O. ; GONÇALVES, DANIEL G. V. ; DUTRA, BRUNO V. ; ALENCASTRO, JOÃO F. DE ; FILHO, FRANCISCO L. DE CALDAS ; MARTINS, LUCAS M. C. E ; \textbf{PRACIANO, BRUNO J. G.} ; ALBUQUERQUE, ROBSON DE O. ; JR, RAFAEL T. DE SOUSA . Design of a Distributed HIDS for IoT Backbone Components. In: 2019 Federated Conference on Computer Science and Information Systems, 2019. org.crossref.xschema.\_1.Title@7bad8b1f, 2019. p. 81.

\item DE MENDONCA, FABIO L. L. ; DA CUNHA, DAYANNE F. ; \textbf{PRACIANO, BRUNO J. G.} ; DA ROSA ZANATTA, MATEUS ; DA COSTA, JOAO PAULO C. L. ; DE SOUSA, RAFAEL T. . P2PIoT: A Peer-To-Peer Communication Model for the Internet of Things. In: 2019 Workshop on Communication Networks and Power Systems (WCNPS), 2019, Brasilia. 2019 Workshop on Communication Networks and Power Systems (WCNPS), 2019. p. 1.

\item BRANDAO, IURE V. ; DA COSTA, JOAO PAULO C. L. ; SANTOS, GIOVANNI A. ; \textbf{PRACIANO, BRUNO J}. G. ; JUNIOR, FRANSCICO C. M. D. ; DE S. JUNIOR, RAFAEL T. . Classification and predictive analysis of educational data to improve the quality of distance learning courses. In: 2019 Workshop on Communication Networks and Power Systems (WCNPS), 2019, Brasilia. 2019 Workshop on Communication Networks and Power Systems (WCNPS), 2019. p. 1.


\item DO NASCIMENTO SILVA, GERSON ; DE CALDAS FILHO, FRANCISCO LOPES ; DOS REIS, VINICIUS ELOY ; \textbf{PRACIANO, BRUNO JUSTINO }; LUSTOSA, JOÃO PAULO ; DE SOUSA JÚNIOR, RAFAEL TIMÓTEO . MODELO DE REDES NEURAIS ARTIFICIAIS EM SUPORTE TECNOLÓGICO À DETECÇÃO DE CARTEIS EM LICITAÇÕES PÚBLICAS. In: Atas da conferência IberoAmericana WWW/Internet 2019, 2019. Atas da conferência Ibero-Americana WWW/Internet 2019. p. 191.

\end{enumerate}

\section{Chapters Description}

This work is divides as follow as: in Chapter 2, the state of the art will be presented, with the goal to support this work, once it is necessary to get a good part of the previous contributions in this research area. In Chapter 3, the theoretical concepts to support the understanding of this work will be presented, since self-drive cars is new trend topic is necessary to describe step by step about this necessary to provide a good theoretical background. In Chapter 4, the proposed framework is shown and the mathematical modelling is provided. In Chapter 5, the results will be discussed around the algorithm performance and the error for object detection position estimation. Finally, in Chapter 6, the conclusions will be discussed regarding the experiments, and some future works will be shown.



	\chapter{State of the art} \label{capitulo2} 
	\chapter{Technical Background}
\label{capitulo3}

In this chapter an overview about the main concepts of this work will be described. The present section is divided in three major sections. In Section \ref{sensors}, the concepts of sensors is defined. In Section \ref{image-processing} covers the technical background about image processing. In Section \ref{ml-ai}, the main concepts regarding machine learning and artificial intelligence are described. In Section \ref{autonomous-vehicles} is defined the major concepts about autonomous vehicles.

\section{Sensors}\label{sensors}

It is possible to have more sensors in autonomous vehicles, but these three sensors are the main in Autonomous vehicles structure. As is positioned in the Figure \ref{fig:autonomous-vehicles} the correct place for the most important sensors.


\begin{figure}[H]
\centering
\includegraphics[scale=0.7]{imagens/image823.png}
\caption{Representation of an autonomous vehicle}
\label{fig:autonomous-vehicles}
\end{figure}


\subsection{Camera}
The camera is a kind of sensor whose allows the car to see the environment using the collected image, in Figure \ref{fig:camera} is shown the standard camera for autonomous vehicles. 

\begin{figure}[H]
\centering
\includegraphics[width=\columnwidth]{imagens/camera.png}
\caption{Representation of a camera of the autonomous vehicle}
\label{fig:camera}
\end{figure}

\subsection{Radar}
\subsection{Lidar}
The lidar is a kind of sensor whose allows the car to see the environment using the collected image, in Figure \ref{fig:lidar} is shown the standard camera for autonomous vehicles. 

\begin{figure}[H]
\centering
\includegraphics[width=\columnwidth]{imagens/lidar.jpg}
\caption{An exemple of a lidar of the autonomous vehicle}
\label{fig:lidar}
\end{figure}

\section{Machine Learning}\label{ml-ai}
The machine learning is approach based on algorithms to create some predictions, these techniques are based on mathematical and computer science, it is possible to apply this in several fields of the science. 
\subsection{Artificial Neural Networks}

The artificial neural networks (ANN) or percpetron were inspired in the human brain. This approach is because the capacity of the human brain to categorize new information. In Figure \ref{fig:ann} is shown an example of the structure of an ANN. Where there is an input array with the processed features for categorization. The next step of the processing is to define the weights for this analysis. The activation function is the main part of this process, because in this step the algorithm will transform the numbers collect by the previous steps and will return only in binary purpose, like $0$ and $1$ in the output layer \cite{goodfellow2016deep}.

\begin{figure}[H]
\centering
\includegraphics[width=\columnwidth]{imagens/ann.png}
\caption{The structure of an ANN}
\label{fig:ann}
\end{figure}

So, mathematically, the neuron output is a function of weighted sum of its inputs. Figure \ref{fig:ann_weight} is shown the mathematical background. Where $f$ is activation function, $w_i$ are the weights, and $\beta$ is the constant input called as bias.


\begin{figure}[H]
\centering
\includegraphics[width=\columnwidth]{imagens/math_ann_bias.png}
\caption{Mathematical representation of ANN with bias}
\label{fig:ann_weight}
\end{figure}


\subsubsection{Activation functions}
There are many different activation functions to use. These are crucial for the ANN characteristics, such as learning ability and computational efforts in terms of training and validation.

\begin{itemize}
    \item Unit step function
    \item Sign function
    \item Identity function
    \item Sigmoid function
    \item Hyperbolic tangent function
    \item Retified linear unit (Relu) 
\end{itemize}

\begin{table}[H]
\label{tab:tab1} 
\caption{Comparison Table for Activation Functions}
\centering
\resizebox{15.3cm}{!}{%
\begin{tabular}{|c|c|c|c|c|c|c|}
\hline
\textbf{\begin{tabular}[c]{@{}c@{}}Activation\\ Function\end{tabular}}  & \textbf{Linear}          & \textbf{Monotonic} & \textbf{Continuous}      & \textbf{\begin{tabular}[c]{@{}c@{}}Derivative \\ Monotonic\end{tabular}} & \textbf{\begin{tabular}[c]{@{}c@{}}Derivative\\ Continuous\end{tabular}} & \textbf{\begin{tabular}[c]{@{}c@{}}Simetric with \\ respect to \\ The Origin\end{tabular}} \\ \hline
{\color[HTML]{000000} Unit Step} & {\color[HTML]{FE0000} x} & {\color[HTML]{009901} \checkmark}& {\color[HTML]{FE0000} x} & {\color[HTML]{FE0000} x} & {\color[HTML]{FE0000} x} & {\color[HTML]{FE0000} x} \\ \hline
Sign & {\color[HTML]{FE0000} x} & {\color[HTML]{009901} \checkmark}& {\color[HTML]{FE0000} x} & {\color[HTML]{FE0000} x} & {\color[HTML]{FE0000} x} & {\color[HTML]{009901} \checkmark} \\ \hline
Identity & {\color[HTML]{009901} \checkmark}& {\color[HTML]{009901} \checkmark}& {\color[HTML]{009901} \checkmark}& {\color[HTML]{FE0000} x} & {\color[HTML]{009901} \checkmark}& {\color[HTML]{009901} \checkmark}\\ \hline
Sigmoid                                                                 & {\color[HTML]{FE0000} x} & {\color[HTML]{009901} \checkmark}             & {\color[HTML]{009901} \checkmark}                   & {\color[HTML]{FE0000} x}                                                 & {\color[HTML]{009901} \checkmark}                                                                   & {\color[HTML]{FE0000} x}                                                               \\ \hline
\begin{tabular}[c]{@{}c@{}}Hyperbolic\\ Tangent\end{tabular}            & {\color[HTML]{FE0000} x} & {\color[HTML]{009901} \checkmark}             & {\color[HTML]{009901} \checkmark}                   & {\color[HTML]{FE0000} x}                                                 & {\color[HTML]{009901} \checkmark}                                                                   & {\color[HTML]{009901} \checkmark}                                                                                 \\ \hline
\begin{tabular}[c]{@{}c@{}}Rectified \\ Linear Unit \\ (ReLU)\end{tabular} & {\color[HTML]{FE0000} x} & {\color[HTML]{009901} \checkmark}              & {\color[HTML]{009901} \checkmark}                   & {\color[HTML]{009901} \checkmark}                                                                   & \begin{tabular}[c]{@{}c@{}}{\color[HTML]{FE0000} x}\\ (at 0)\end{tabular}                       & {\color[HTML]{FE0000} x}                                                               \\ \hline
\end{tabular}}
\end{table}

In this work, Relu is the activation function chosen activation. Its mathematical definition is show in (\ref{eq:relu}). 


\begin{equation}
\label{eq:relu}
    f(x) = \mathrm{R}(x) = \mathrm{max}(0,x)
\end{equation}


The graphic visualization of the function Relu is shown in Figure \ref{fig:relu}, where this function return $0$ if the number from the weighted sum is below than $0$, or return $x$ if the previous value is over than $0$.


\begin{figure}[H]
\centering
\includegraphics[scale=0.3]{imagens/relu_corrected.png}
\caption{The behavior of the Relu}
\label{fig:relu}
\end{figure} 

These are the important characteristics of this activation function.

    \begin{itemize}
        \item Widely used in deep networks
        \item Pros: nonlinear, monotonic, derivative monotonic, and fast convergence
        \item Cons: not continuously differentiable at zero, i.e. issues with gradient descent around origin.
    \end{itemize}

\subsubsection{Layered Neural Networks}

The quint essential example of a deep learning model is the multilayer perceptron (MLP). A MLP is just a mathematical function mapping some set of input values to output values. The function
is formed by composing many simpler functions \cite{goodfellow2016deep}.


These are the quint essential deep learning models. The goal
of a feedforward network is to approximate some function $f*$. For example, for a classifier, $y = f*(x)$ maps an input x to a category y. A feedforward network
defines a mapping $y = f (x; \theta)$ and learns the value of the parameters $\theta$ that resultin the best function approximation.

These models are called feedforward because information flows through the function being evaluated from $x$, through the intermediate computations used to define $f$, and finally to the output $y$. There are no feedback connections in which outputs of the model are fed back into itself.



\subsection{Convolutional Neural Networks}


The Convolutional Neural Networks (CNN) are a specialized kind of neural network for processing data that has a known \cite{lecun1995convolutional}. For example in autonomous vehicles domain, this approach is several used for object detection and object identification. This name  indicates that the network employs a mathematical operation called
convolution. 

A CNN coarsely scans the image for features (in lower dimension space), pools possible patterns, then inspect those patterns in detail with its fully connected
subnetworks, generating their classifications. In Figure \ref{fig:cnn_car} is defined the full process of this neural network. Where there are three other importants is steps: Convolutional layer is defined in Subsection \ref{sub:conv}. The pooling layer is introduced in the Subsection \ref{sub:pooling}. The fully connected layers is introduced in Subsection \ref{sub:fully}.

\begin{figure}[H]
\centering
\includegraphics[width=\columnwidth]{imagens/Full_Process.png}
\caption{Full process of a convolutional neural network}
\label{fig:cnn_car}
\end{figure}


\subsubsection{Convolutional Layer}
\label{sub:conv}

When the convolutional layer is introduced, common inputs are tridimensional matrix with height and width defined accordingly with the image dimensions and determined by the amount of colors. In general the images use three color channels, Red-Green-Blue (RGB) as is shown in Figure \ref{fig:rgb}.

\begin{figure}[H]
\centering
\includegraphics[scale=0.35]{imagens/rgb_representation.png}
\caption{Representation of the colors of the input image}
\label{fig:rgb}
\end{figure}


The convolutions work as filters that seem little squares and they are slipping through whole image and capturing the most important parts. For example in Figure \ref{fig:bias}, an image with $NxNX3$ and a filter with the $MxMX3$, where the main different is that each result is then summed, along with the Bias ($\beta$) value, to be then passed to the activation function. And in the end of the process generates a new matrix called as feature map or activation map.


\begin{figure}[H]
\centering
\includegraphics[scale=0.35]{imagens/three_dim_conv_2.png}
\caption{Representation of the convolution process}
\label{fig:bias}
\end{figure}




\subsubsection{Pooling and Upsampling}\label{sub:pooling}

A pooling layer is necessary to simplify the information from the previous layer. As happens with the convolution layer, it is choose an unit area, for example $2x2$ to slicing for the whole output information from the previous step. To brief, if the information from the previous layer was $4x4$, the output from process of pooling will be $2x2$. Nevertheless, the most used method is maxpooling, where the biggest number in the matrix is passed to the next step, this data summarization is used to reduce the amount of weights and to avoid overfitting. In Figure \ref{fig:pooling} is shown the maxpooling process.

\begin{figure}[H]
\centering
\includegraphics[scale=0.35]{imagens/max_pooling.png}
\caption{Representation of the maxpooling process}
\label{fig:bias}
\end{figure}




\subsubsection{Auto-encoders}
\subsubsection{Training}
\subsubsection{Losses}
\subsubsection{Stochastic Gradient Descent}
\subsubsection{Weight Initialization}
\subsubsection{Error Backpropagation}

\subsubsection{Fully Connected Layers}\label{sub:fully}
\subsubsection{Regularization}
\subsubsection{Data Augmentation}



\section{Autonomous Vehicles} \label{autonomous-vehicles}
\subsection{What are autonomous vehicles}
\subsection{History and current research}
\subsection{Challenges}
\subsection{Levels	of	Automation}
\subsection{Effects of autonomous vehicles in the society} 
	
\chapter{Proposed Framework}
\label{capitulo4}


\section{Proposals}

\subsection{One camera with object calibration}
\begin{figure}[H]
\centering
\includegraphics[width=\textwidth]{imagens/proposal1.png}
\caption{Proposal using only one camera with object calibration, this firts approach is based on \cite{8678911}}
\label{fig:proposal1}
\end{figure}


\subsection{Proposal 2 – One camera with known map}
\begin{figure}[H]
\centering
\includegraphics[width=\textwidth]{imagens/proposal2.png}
\caption{Proposal using Proposal 2 – One camera with known map}
\label{fig:proposal2}
\end{figure}


\subsection{Proposal 3 – Multicameras}
\begin{figure}[H]
\centering
\includegraphics[width=\textwidth]{imagens/proposal3.png}
\caption{Proposal using multicameras}
\label{fig:proposal3}
\end{figure}



\section{Inverse Perspective Mapping }
Inverse perspective mapping is a mathematical technique that remove the effects of distortion of a picture when transforming the perspective of the image to another perspective. In spite of disparity mapping, inverse perspective mapping method requires only one camera and this method can't provide depth information directly ~\cite{Tuohy2010}.

Camera must be located in front of the car with an angle of \(\theta\) to down. Figure \ref{fig:ImageRelationSystem} shows the setup.

\begin{figure}[h]
\centering
\includegraphics[scale=0.5]{imagens/Inverse Perspective Mapping.JPG}
\caption{Image coordinate system in relation to world coordinate
system.}
\label{fig:ImageRelationSystem}
\end{figure}
\par


This setup was selected based on solution of \cite{Wongsaree2018}, the mathematical background is to create top-down view, the surface road point is known as $(X_w,Y_w,Z_w)$
that projects to the image plane $(u,v)$ is a must. As disrupted in Figure \ref{fig:ImageRelationSystem}. For rotatation angle $(\theta)$, which is angle between camera and the surface, the IPM equation is based on \cite{7759904} and is shown is Equation \ref{eq:eq1}:

\begin{equation}
    (u,v,1)^T = K\cdot T \cdot K (X_w,Y_w, Z_w,1)^T
    \label{eq:eq1}
\end{equation}

where R is the rotation matrix given in the equation \ref{exp2}.
\begin{equation} \label{exp2}
R=
\begin{bmatrix}
1 & 0 & 0 & 0\\
0 & \cos{\theta} & -\sin{\theta} & 0\\
0 & \sin{\theta} & \cos{\theta} & 0\\
0 & 0 & 0 & 1
\end{bmatrix}
\end{equation}
\par

T is the translation matrix given in the equation \ref{exp3}. Where h means the height of the position of the camera.
\begin{equation} \label{exp3}
T=
\begin{bmatrix}
1 & 0 & 0 & 0\\
0 & 1 & 0 & 0\\
0 & 0 & 1 & \frac{-h}{\sin{\theta}}\\
0 & 0 & 0 & 1
\end{bmatrix}
\end{equation}


\par
K is the camera parameter matrix given in the Equation \ref{exp4}. Where $f$ is the focal length of the camera, $s$ is the skew parameter and $u_0, v_0$ are the center of the pixel of desired image size. 
\begin{equation} \label{exp4}
K =
\begin{bmatrix}
f & s & u_0 & 0\\
0 & f & v_0 & 0\\
0 & 0 & 1 & 0\\
\end{bmatrix}
\end{equation}

The Equation \ref{exp4} can be replaced using the real parameters of this test scenario and these parameters are $f = 2.92 mm, s=0, u_0=240, v_0=160$. Replacing the Equations \ref{exp2},\ref{exp3}, \ref{exp4} into the initial Equation \ref{eq:eq1}, achieving the new Equation \ref{eq:eq2}.

\begin{equation}
    \begin{bmatrix}
u\\ 
v\\ 
1
\end{bmatrix}
=\begin{bmatrix}
P_{11} & P_{12} & P_{13} & P_{14}\\ 
P_{21} & P_{22} & P_{23} & P_{24}\\ 
P_{31} & P_{32} & P_{33} & P_{34}
\end{bmatrix}
\begin{bmatrix}
X_w\\ 
Y_w\\ 
Z_w\\
1
\end{bmatrix}
\label{eq:eq2}
\end{equation}

where the matrix P was gotten from product between K, T, and R. As is only necessary to evaluate the position of the road, so the coordinate $Y_w$ can be equal to 0, so simplifying the Equation \ref{eq:eq2}, so it is given by Equation \ref{eq:eq3}.

\begin{equation}
    \label{eq:eq3}
    \begin{bmatrix}
u\\ 
v\\ 
1
\end{bmatrix}
=\begin{bmatrix}
P_{11} & P_{12}  & P_{14}\\ 
P_{21} & P_{22}  & P_{24}\\ 
P_{31} & P_{32}  & P_{34}
\end{bmatrix}
\begin{bmatrix}
X_w\\ 
Z_w\\
1
\end{bmatrix}
\end{equation}

Based on the Equations above, it is possible to infer the Equation \ref{eq:eq4} for compute the distance from the camera until the object. 

\begin{enumerate}
    \item Calculating average intensity in row direction from bottom row up to top row
    \item The average intensity of each row is compared with the threshold level (obtained from the experimental) which is 50. The starting position of an object is indicated if the average intensity in that row is greater than 50 and the order of that row is stored in a parameter p.
    \item The distance between object and vehicle is therefore calculated using a linear equation given in \ref{eq:eq4}.
\end{enumerate}

\begin{equation}
    \label{eq:eq4}
    d = ap+b
\end{equation}

where $d$ is distance between camera and object and vehicle in meter, $p$ is the order of the row that object is detected and $a, b$ are constants.

 
\section{Framework Architecture} 

\begin{figure}[H]
\centering
\includegraphics[scale=0.6]{imagens/diagram.png}
\caption{Architecture approach of framework}
\label{fig:framework}
\end{figure}

\begin{figure}[H]
\centering
\includegraphics[width=\textwidth]{imagens/Network Behavior.png}
\caption{Architecture approach of framework}
\label{fig:networkBehavior}
\end{figure} 
	\chapter{Results}
\label{capitulo5}

In this chapter the results collected from the Chapter 4 will be discussed and analyzed. It is important to freeze that only the approach 3 from \ref{sub:3} was implemented into the framework, due to the goal was to work with multiple cameras and multi-view perspective.

The algorithms and the simulations was performed in a computer with this follow configuration:

\begin{itemize}
    \item Operational System Ubuntu 18.04
    \item CPU Intel core i7 7700HQ 2.80 GHz
    \item 32 GB memory RAM
    \item GPU Nvidia Geforce GTX 1050 Ti - 4 GB
\end{itemize}


The idea of the framework is to perform the tests on Audi test track as shown in Figure \ref{fig:test_track}, but in this work only the proof of concept of the algorithms was performed. 

\begin{figure}[H]
\centering
\includegraphics[scale=0.3]{imagens/testtrack.jpg}
\caption{Audi test track in eagle's view}
\label{fig:test_track}
\end{figure}


The algorithm for object detection was performed over the parking lot of the company EFS GmbH as shown in Figure \ref{fig:park}.


\begin{figure}[H]
\centering
\includegraphics[scale=0.5]{imagens/park.JPG}
\caption{Position of the cars on the parking lot}
\label{fig:park}
\end{figure}

\section{Description of the test scenario}

The test scenario was built three times using different perspectives, for the first test where was necessary just to use the camera calibration perspective was built using only one camera and the height of this camera does not matter to compute the distance, only a known distance and dimensions of a known object to adjust this algorithm.

For the second scenario some photos was taken and labeled with the known metrics to support the training step and return a good result according with reality. 

The last approach and used in this work to built the framework was used following some principles as the cameras should be mounted at the same level, the same horizontal position,  the stream must be captured at the same time of the camera and sent to the same control center to process this data. 

The used cameras were GoPro 5 which allow to built a wireless network with many cameras 


\section{Results with camera calibration}

The results provided from the approach 1 from \ref{sub:1} was implemented using Python 3.7 and Opencv3, and the position was computed based the camera calibration. In (\ref{eq:focal_distance}) is defined the distance used to calibrate the camera, this step has been done with a measurer tape and a piece of paper ($21.59$ cm x  $27.94$ cm) and this was positioned $60.96$ cm in front of the camera to take the photo.

\begin{equation}
    \label{eq:focal_distance}
    F = \frac{P\cdot D}{W}
\end{equation}

where W is the width of the piece of the paper, in this case is $27.94$ cm, D is distance from the piece of paper to the camera, and P is the measure of the paper in pixels, this is taken from image. Applying the (\ref{eq:focal_distance}), the focal length (F) is $541.09$ pixels.





The results achieved with this technique is shown in Table \ref{tab:output_calibrate}. 

\begin{table}[H]
\centering
\caption{Measurements achieved with camera calibration algorithm}
\begin{tabular}{l|l} 
\toprule
Car &  Measurements      \\
\#1   & 4.05        \\
\#2   & 10.67       \\
\#3   & 15.52       \\
\#4   & 19.55       \\
\#5   & 20.08       \\
\bottomrule
\end{tabular}
\label{tab:output_calibrate}
\end{table} 



\section{Results with known map}
In this section is detailed the results of the proposal with known map, this approach was performed using the neural network from \ref{sub:2} and the data provided from KITTI Dataset \cite{geiger2013vision}, it is important to freeze, this approach was used only for comparison method and not to be used on the final framework. 


The results collected in this section are very useful, as several companies and other universities are releasing the dataset as open source. With this, there is a need to understand how to manipulate a set of images, and in some cases even data from RADAR and LIDAR are available.

The main ideia of this test was to get the information from the object detection performed with the algorithm using boundary boxes approach and collect this known information as input in the neural network and start to predict a distance from the camera until the object. Where in Figure \ref{fig:output} is shown an example of the known KITTI dataset, it serves just as motiviation for this work, the final results were not performed over this dataset. 

\begin{figure}[H]
\centering
\includegraphics[width=\textwidth]{imagens/ouput.png}
\caption{Output results from framework using single stereo camera and known map}
\label{fig:output}
\end{figure}

For this approach the output achieved from Table \ref{tab:output_table} was used for the first interaction and estimate the distance as shown in Table \ref{tab:output_2}. These results are a little bit similar with the reality because the map is already known and the measurement step of these structures was performed and the labelling part as well.

\begin{table}[H]
\centering
\caption{Measurements achieved with camera and known map}
\begin{tabular}{l|l} 
\toprule
Car &  Measurements      \\
\#1   & 4.43        \\
\#2   & 10.98       \\
\#3   & 16.01       \\
\#4   & 18.99       \\
\#5   & 22.18       \\
\bottomrule
\end{tabular}
\label{tab:output_2}
\end{table} 


\section{Results with multicameras and proposed framework}


The algorithm of the proposed framework predict the objects of the whole scenario in 28 ms and has identified 9 cars on the image as show in Figure \ref{fig:park_predict} and in Table \ref{tab:accuracy} is shown the accuracy of each prediction for each car. The output accuracies from the algorithm are shown in Table \ref{fig:framework_predict}, the low accuracies are related with the partial occluded objects, such as the car 2 and car 6. 

The perspective of the distance, this algorithm computed this using the Inverse Perspective Mapping (IPM) combined with Yolo Algorithm, where the perspective of distance was fused on the last fully connecte layer. In Table \ref{tab:output_framework} is shown the results achieved from the perspective using two cameras. 

\begin{table}[H]
\centering
\caption{Measurements achieved with multicameras}
\begin{tabular}{l|l} 
\toprule
Car &  Measurements      \\
\#1   & 4.40        \\
\#2   & 11.05       \\
\#3   & 16.01       \\
\#4   & 19.92       \\
\#5   & 24.08       \\
\bottomrule
\end{tabular}
\label{tab:output_framework}
\end{table} 
 

In Figure \ref{fig:park_predict} was used just the model that contains the object recognition and detection using a single image and not as stream, this test was performed to see the accuracy of the algorithm in this purpose and identify how far is possible to detect the objects in this scenario, it is necessary to note that this test previewed only 5 cars to detection and the algorithm detect the whole cars in the scenario, each accuracy for each car of the test is available in Table \ref{tab:accuracy}.


\begin{figure}[H]
\centering
\includegraphics[scale=0.3]{imagens/predictions.jpg}
\caption{Output image with predictions }
\label{fig:park_predict}
\end{figure}



\begin{table}[H]
\centering
\caption{Accuracy of the proposed framework in object detector and classification}
\begin{tabular}{c|c}
\hline
Predicted Label & Accuracy \\ \hline
Car             & 91\%     \\ \hline
Car             & 31\%     \\ \hline
Car             & 96\%     \\ \hline
Car             & 94\%     \\ \hline
Car             & 95\%     \\ \hline
Car             & 98\%     \\ \hline
Car             & 47\%     \\ \hline
Car             & 97\%     \\ \hline
Car             & 98\%     \\ \hline
\end{tabular}
\label{tab:accuracy}
\end{table}



In Figure \ref{fig:framework_predict} is shown the frontend of the application of this work, it was developed using Python and Javascript to allow the browser communicate with the model. The Python module is composed by the libraries called sockets, and it permits the communication over many cameras and share the stream information between the cameras and the command central, it was possible because the cameras used in the tests have internal wireless network. And based on it a script with a proxy function was developed to take care of this behavior. 

The backend side of this project is described in the Appendix \ref{ap:app} where there are the informations about the pre-processment step, training step, object detection, and the frontend scenario as well. The network was built using the framework Pytorch \cite{paszke2019pytorch} this script permits the abstraction of the Darknet to perform the object detection and the object recognition, and with all of this information the step to predict the distance was combined to show the output. 

The camera 1 of the Figure \ref{fig:framework_predict} is used only to perform the boundary boxes detections and the camera 2 is used to perform the distance prediction. This solution was embedded in a module to work as a controller of this framework.

The experiments led us also to measure the rapidity of the multicameras method by computing the number of frames treated per second.  The Fig. 9 shows that the method could treat up to 23 frames per second. the average of frames per second through all the experiments is 20.57 frames per second which is enough for real time treatments. 

\begin{figure}[H]
\centering
\includegraphics[scale=0.8]{imagens/output_framework.png}
\caption{Output image with predictions }
\label{fig:framework_predict}
\end{figure}



\section{Validation}

For validation purpose, it was used a commercial laser measurement as shown in Figure \ref{fig:laser_meas}, this model is known as Bosch DLE 40 Professional {\tiny{\textregistered}}. 



\begin{figure}[H]
\centering
\includegraphics[scale=0.3]{imagens/trena.jpg}
\caption{Commercial laser measurer}
\label{fig:laser_meas}
\end{figure}

The instrumental error rate is $\pm 1.5 mm$, thereby we repeated the measure three times and computed the mean, and standard deviation as well. In Table \ref{tab:tab_measure} is shown the measurements with the camera positioned at $2.01$ m from the ground. And in Figure \ref{fig:park} is shown the position of the cars along the parking lot. The reason to measure three times is because in the manual is written that in sunny days can bias the measurement. 



\begin{table}[H]
\centering
\caption{Measurements collected with a commercial measurer}
\begin{tabular}{l|l|l|l|l} 
\toprule
Car & First measure (m) & Second measure (m) & Third measure (m) & Mean (m) \\
\#1   & 4.25          & 4.47           & 4.51           & 4.41 \\
\#2   & 11.01         & 11.21          & 11.11          & 11.11\\
\#3   & 16.12         & 16.35          & 16.26          & 16.24\\
\#4   & 19.63         & 19.69          & 19.66          & 19.66\\
\#5   & 23.08         & 23.18          & 23.01          & 23.09\\
\bottomrule
\end{tabular}
\label{tab:tab_measure}
\end{table} 


For purpose to comparison and to see the differences between the approaches and the real distance computed with the tool from Figure \ref{fig:laser_meas}. The Table \ref{tab:total} was built to facilitate the visualization of these outputs. 


\begin{table}[H]
\centering
\caption{Comparison between the algorithms and real data}
\begin{tabular}{l|l|l|l|l} 
\toprule
Car & Camera Calibration (m) & Known Map (m) & Multicameras (m) & Real distance (m) \\
\#1   & 4.05          & 4.43           & \textbf{4.40 }          & 4.41 \\
\#2   & 10.67         & 10.98          & \textbf{11.05 }         & 11.11\\
\#3   & 15.52         & \textbf{16.01}          &\textbf{ 16.01 }         & 16.24\\
\#4   & \textbf{19.55}         & 18.99          & 19.92          & 19.66\\
\#5   & 20.08         & 22.18          & \textbf{24.08}          & 23.09\\
\bottomrule
\end{tabular}
\label{tab:total}
\end{table} 

For to analyze the difference between the implemented methods, it is necessary to examine the error between the real value. For this a Python script was implement using Pandas Library \cite{mckinney2011pandas} to take care of the collected data. The error analysis was implemented to make it easier the visualization of the best applied technique. 

The pre-processed samples are used as input for the algorithms under text, resulting in estimations of the true position value. Results are expressed in terms of the RMSE, given by (\ref{eq:rmse}):
%
\begin{equation} \label{eq:rmse}
\text{RMSE}(f, \hat{f}) = \sqrt{\frac{1}{n_\text{samples}} \sum_{i=0}^{n_\text{samples} - 1} (f_i - \hat{f}_i)^2},
\end{equation}
%
calculated for each estimator $\hat{f}_i$, referenced either from the measurement tool of Figure \ref{fig:laser_meas} true value $f_i$ as read at the end of each measurement.

In Figure \ref{fig:rmse} is shown a chart with the three proposed techniques using (\ref{eq:rmse}), where is possible to see in certain points the multicameras approach was better and the camera calibration was the worse approach for this case. 

\begin{figure}[H]
\centering
\includegraphics[scale=0.6]{imagens/plot.png}
\caption{RMSE of estimated estimate position for each algorithm for different detected car, referenced to values measured by the commercial laser measurer}
\label{fig:rmse}
\end{figure} 
    \chapter{Conclusion}
\label{capitulo6}


In the next years, autonomous vehicles will be a new reality, and this moment the topics regarding object detection and machine learning are being the hot trends in the computer science domain. And with this, it is necessary to improve the methods of computer vision to improve this related technology. It is possible to apply this algorithm to detect other kinds of objects and perform distance measurement as well. 

A Real-time distance measurement method with multi-cameras for object detection on the roads is introduced in this work. The utilized method is based on using multi-cameras, which are two cameras mounted in the same horizontal position and displaced vertically by a predefined distance (the base). To measure the distance to objects, a vehicle detection method is performed first following two steps: hypothesis generation and hypothesis verification. 

% Several methods apply the detection task on both images, which is time-consuming. However, in this paper the vehicles are detected first in only one camera then similar vehicles are detected on the other camera using a stereo matching method. After detecting and matching the same vehicles in both cameras, the distance measurement method based on the distance between the two cameras, the position of vehicles in both cameras and certain geometric angles, is performed. Although the method is based on a relatively simple algorithm, the distance is measured accurately. Furthermore, a comparison between the proposed method and some other methods from literature was performed to evaluate the proposed method, where it showed that despite the simplicity of the proposed method, it measures the distance with high accuracy. The proposed method may be used to perform several tasks in several systems such as computing safety distance between vehicles and vehicle speed and it may also be used to measure objects distances regardless of their types by simply changing the detection algorithm.

This work also makes a comparative approach between the other several state-of-the-art techniques algorithms to perform distance measurement to choose the best and faster technique. 

One of the big problems on camera arrays is the moment to put in the correct angle and high level, due to this a small filter to clear this threshold is necessary to be implemented and remove these noises. The part of calibration in the proposed technique 1 is a little bit difficult, because it is necessary to remember the tool error and the measurement error, and with this is possible to calibrate in the wrong way. 

It was also proved that the multi-cameras perspective combined with a high order mathematical technique is better for this work, where it is possible to divide the work of the detection into two different or more cameras. However for our motivation and based on the current literature the angle between the camera is +30 degrees, with this will allow us to cover a big part of the scenario.   

\section{Future works}

According to all results collected in this work, it is possible to suggest the following approaches:

\begin{itemize}
    \item Collect data and label data to perform predictions for a specific scenario.
    \item Combine data from multiple sensors, such as LIDAR or RADAR using data fusion approaches to increase the accuracy of the measurement. 
    \item Apply other kinds of algorithms, like EfficienceDet or other novel algorithms for object detection and distance measurement as well.
    \item Use the proposed test in the real scenario because the algorithm was just performed in controlled sites. 
    \item Perform other mathematical approaches to reduce the dimensionality of the data.
\end{itemize}
% 	\bibliographystyle{apalike}	
    \nocite{*}
	\bibliographystyle{abnt-num} % estilo bibliográfico ABNT numérico
% 	\bibliographystyle{abnt-alf} % estilo bibliográfico ABNT alfabético
% 	\bibliographystyle{sbc}  % estilo bibliográfico da Sociedade Brasileira de Computação (SBC)
	
	%\renewcommand{\bibname}{REFERÊNCIAS BIBLIOGRÁFICAS} %Define o Caption da seção de bibliografia
	%\addcontentsline{toc}{chapter}{REFERÊNCIAS BIBLIOGRÁFICAS}
	
	%não pode ter espaço entre os nomes dos arquivos bib
	\bibliography{referencias}
	\appendix
	\chapter{Appendix}

\section{Code to control the app}]\label{ap:app}
\lstinputlisting[language=Python]{code/app.py}

\section{Code to detect the bounding boxes}\label{ap:bbox}
\lstinputlisting[language=Python]{code/bbox.py}

\section{Code to control the camera}\label{ap:camera}
\lstinputlisting[language=Python]{code/camera.py}

\section{Abstraction of Darknet in Pytorch}\label{ap:darknet}
\lstinputlisting[language=Python]{code/darknet.py}

\section{Code for data preprocessing}\label{ap:preprocess}
\lstinputlisting[language=Python]{code/preprocess.py}

\section{Base Template}\label{ap:template1}
\lstinputlisting[language=html]{code/base.html}

\section{Index Template}\label{ap:template2}
\lstinputlisting[language=html]{code/index.html}







\end{document}

